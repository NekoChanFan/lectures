\documentclass[oneside,final,14pt]{extreport}
\input{/home/nekochan/lectures/templates/preamble.tex}

\title{\Huge{Компьютерная дингвистика}\\ Конспекты по предмету}
\author{\huge{Милов Данила}}
\date{}

\begin{document}
\maketitle
\newpage
\tableofcontents
\newpage
\chapter{Конечные преобразователи}
\section{Конечные преобразователи}
\dfn{Конечный преобразователь}{Шестёрка $S=(Q,\Sigma,\Delta,q_0, \delta,F)$, где:\\
Q --- множество состояний,\\
$\Sigma$ --- входной алфавит,\\
$\delta\subset Q\times\Sigma^*\times\delta \times Q$ --- конечное множество переходов(программа),\\
$q_0$ --- начальное состояние,\\
$F \subseteq Q$ --- множество закл. состояний}

\dfn{Конфигурация}{
  Тройка $(q,u,v)$, где $q\in Q, u\in \Sigma^*, v\in \Delta^*$
}
\dfn{Переход за один шаг($\vdash$)}{
  $(q,u,v)\vdash_S(p,x,y), \text{ если существует }w\in\Sigma^*,t\in\Delta^*:\\
  u=wx,y=vt$ и $q,w\rightarrow p,v \in \delta$
}
\dfn{Отношение, вычисляемое S}{
  $R(S)= \{(x,y):(q_0,x,\varepsilon) \vdash^*_S (q_f,\varepsilon,y)\text{ для некоторого }q_f\in F\}$
}
\dfn{Детерминированный КП}{
Шестёрка $S=(Q,\Sigma,\Delta,\delta,q_0,F),\text{ где }\delta: Q\times\Sigma\rightarrow Q\times\Delta^*$}

\qs{Построить КП}{$\Sigma = \{a,b,c,\$\}$\\
Заменить все чётные символы b на c(нумерация с 1).}
\begin{figure}[H]
  \begin{center}
    \includegraphics[width=0.5\textwidth]{1.jpg}
  \end{center}
\end{figure}

\qs{Построить КП}{$\Sigma = \{a,b,c,\$\}$\\
Стереть все символы a, стоящие непосредственно после символов b.}
\begin{figure}[H]
  \begin{center}
    \includegraphics[width=0.5\textwidth]{2.jpg}
  \end{center}
\end{figure}

\qs{Построить КП}{$\Sigma = \{a,b\}$\\
Удалить из x все блоки символов a чётной длины.}
\begin{figure}[H]
  \begin{center}
    \includegraphics[width=0.5\textwidth]{3.jpg}
  \end{center}
\end{figure}

\nt{Домашнее задание:\\№134(в,г),133,137}

\section{Морфологический анализ}
\noindent Задачи морфологического анализа:
\begin{enumerate}
  \item Дано слово, поставить его в нужную форму.
  \item Дано слово, найти его начальную форму.
\end{enumerate}

\mprop{Рассматриваемые характеристики}{
  \begin{itemize}
    \item N --- существительное(noun)
    \item V --- глагол(verb)
    \item SG --- единственное число(singular)
    \item PL --- множественное число(plural)
  \end{itemize}
}
\ex{}{
  cat+N+PL $\Rightarrow$ cats\\
  cats $\Rightarrow$ cat+N+PL
}

\dfn{Морфологические правила}{
  Определяют, как образуются формы.\\
  Пример: английский язык, множественное лицо = ед. число + s.
}

\dfn{Орфографические правила}{
  Определяют, как меняется написание.\\
  Пример --- англ. язык, мн. число:\\
  Слово заканчивается на s $\rightarrow$ вставить е.\\
  Слово заканчивается на y $\rightarrow$ заменить на iе.\\
  Слово является исключением $\rightarrow$ особый случай.
}

\dfn{Основа}{<<Главная часть>> слова.}

\dfn{Аффексы}{Части слова, приписываемые к основе:
\begin{enumerate}
  \item Префиксы: пишутся спереди.
  \item Суффиксы: пишутся сзади.
  \item Инфиксы: пишутся внутри.
  \item Циркумфиксы: спереди и сзади.
\end{enumerate}}

\nt{Для морфологического анализатора необходимы:
\begin{enumerate}
  \item Словарь --- список основ и аффексов.
  \item Морфотактика --- правила объединения морфем.
  \item Орфографические правила.
\end{enumerate}}

\newpage
\ex{Автомат, проверяющий правильность определения множественного числа}{
  \textbf{reg-noun} --- правильные существительные;\\
  \textbf{irreg-sg-noun} --- единственное число <<неправильных>> существительных;
  \textbf{irreg-pl-noun} --- множественное число <<неправильных>> существительных;\\
  \textbf{plural(-s)} --- s.
  \begin{figure}[H]
    \begin{center}
      \includegraphics[width=0.5\textwidth]{4.jpg}
    \end{center}
  \end{figure}
}

\dfn{Префиксное дерево}{Есть множество слов $S=\{S_1,\dots,S_n\}$. \begin{enumerate}
  \item Рёбра дерева помечены символами.
  \item Рёбра, выходящие из одной вершины, помечены разными символами.
  \item На пути из корня в листья написаны слова из $S$.
  \item Для каждого $S_i$ существует вершина $v$ такая, что на пути из корня в $v$ написано $S_i$($v$ отмечены $i$).
\end{enumerate}}

\newpage
\ex{Префиксное дерево}{
  Префиксное дерево для слов: алгебра, алгоритм, гомоморфизм, гомотопия, гомеоморфизм, шар, шары.
  \begin{figure}[H]
    \begin{center}
      \includegraphics[width=0.5\textwidth]{5.jpg}
    \end{center}
  \end{figure}
}

\ex{Автомат, распознающий множественное число слов}{
  \begin{figure}[H]
    \begin{center}
      \includegraphics[width=0.5\textwidth]{6.jpg}
    \end{center}
  \end{figure}
}

\newpage
\ex{Автомат для распознавания прилагательных(1)}{
  \textbf{adj-root} --- корень прилагательного\\
  \textbf{un-, -er, -est, -ly} --- аффексы слов.
  \begin{figure}[H]
    \begin{center}
      \includegraphics[width=0.5\textwidth]{7.jpg}
    \end{center}
  \end{figure}
}
\ex{Автомат для распознавания прилагательных(2)}{
  \textbf{adj-root$_1$} --- корень прилагательного, который может употребляться с приставками\\
  \textbf{adj-root$_2$} --- корень прилагательного, который не может употребляться с приставками\\
  \textbf{un-, -er, -est, -ly} --- аффексы слов.\\
  Это недетерминированный вариант автомата(угадываем переход по пустому слову в $q_1$ или $q_3$).
  \begin{figure}[H]
    \begin{center}
      \includegraphics[width=0.5\textwidth]{8.jpg}
    \end{center}
  \end{figure}
}

\mprop{Три уровня конечного преобразователя}{
  \begin{enumerate}
    \item Лексический: слово и его признаки.(
      \begin{tabular}{|c|c|c|c|c|}
        \hline f&o&x&+N&+PL\\\hline
      \end{tabular}
      )
    \item Промежуточный: морфемы и доп.метки(
      \begin{tabular}{|c|c|c|c|c|}
        \hline f&o&x&$\hat{\phantom{a}}$&s\\\hline
      \end{tabular}
      )
    \item Поверхностный: слово(
      \begin{tabular}{|c|c|c|c|c|}
        \hline f&o&x&e&s\\\hline
      \end{tabular}
      )
  \end{enumerate}
}

\ex{Преобразователь слов с лексического уровня в промежуточный}{
  Обозначения:\\
  $\alpha:\beta$ --- читаем $\alpha$, пишем $\beta$,\\
  $\alpha$ --- читаем $\alpha$ и печатаем $\alpha$.
  \begin{figure}[H]
    \begin{center}
      \includegraphics[width=0.5\textwidth]{9.jpg}
    \end{center}
  \end{figure}
}

\end{document}
